
%!TEX root = main.tex
%\BDOC
%§ex
% Here are commands related to creating exercises
%:-
\makeatletter
%Macro containing exercise number
\def\exerciseNr{0}
%:=\problem{text}
%: This command will print out a problem header. For example \problem{1}
%: prints a nice big header \!textbf{Problem 1} You can do a star (*) after \problem to
%: prevent it from showing in the table of contents
%:-
%:=\problem*{text}
%: Does the same as \problem, but does not add the problem to the table of contents
%:-
\gdef\problem{\@ifnextchar*{\expandafter\@problem\@secondoftwo}{\@problemTOC}}
\gdef\@problemTOC#1{
  \addcontentsline{toc}{section}{\@tr{Problem} #1}
  \@problem{#1}
}
\newcommand{\@problem}[1]{
% Update exercise number
  \def\exerciseNr{#1}
% Move margins
  \begin{addmargin}{-1.5em}
% Problem and problem number
    {\normalfont\Large\bfseries \@tr{Problem} #1}
% Move margins back
  \end{addmargin}
}
%:=\pproblem{text}
%: This command will print out a part problem header based on what problem you are on.
%: For example if you already have done \problem{1}, then \pproblem{a}
%: prints a nice big header \!textbf{(1a)}. You can do a star (*) after \pproblem to
%: prevent it from showing in the table of contents
%:-
%:=\pproblem*{text}
%: Does the same as \pproblem, but does not add the part problem to the table of contents
%:-
\gdef\pproblem{\@ifnextchar*{\expandafter\@pproblem\@secondoftwo}{\@pproblemTOC}}
\gdef\@pproblemTOC#1{
  \addcontentsline{toc}{subsection}{\exerciseNr #1}
  \@pproblem{#1}
}
\newcommand{\@pproblem}[1]{
% Make some space above
  \vspace*{1em}
% Write out problem number and letter
  \hspace*{-0.5em}{\normalfont\Large\bfseries \exerciseNr #1)}
}
\makeatother
