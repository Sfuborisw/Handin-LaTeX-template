
%!TEX root = oblig1.tex
%\BDOC
%§ex
% Here are commands related to creating exercises
%\EDOC
\makeatletter
%Macro containing exercise number
\def\exerciseNr{0}
% \BDOC
% \problem{text}
% This command will print out a problem header. For example \problem{1}
% prints a nice big header \!textbf{Problem 1}
% \EDOC

\newcommand{\problem}[1]{
% Update exercise number
  \def\exerciseNr{#1}
% Move margins
  \begin{addmargin}{-1.5em}
% Problem and problem number
    {\normalfont\Large\bfseries \@tr{Problem} #1}
% Move margins back
  \end{addmargin}
}
% \BDOC
% \pproblem{text}
% This command will print out a part problem header based on what problem you are on.
% For example if you already have done \problem{1}, then \pproblem{a}
% prints a nice big header \!textbf{(1a)}
% \EDOC
\newcommand{\pproblem}[1]{
% Make some space above
  \vspace*{1em}
% Move margins
  \begin{addmargin}{-0.5em}
% Write out problem number and letter
    {\normalfont\Large\bfseries \exerciseNr #1)}
% Move margins back
  \end{addmargin}
% Make some space below
  \vspace*{1em}
}
\makeatother
